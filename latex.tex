\documentclass[conference]{IEEEtran}

\usepackage{graphicx}
\usepackage{amsmath}
\usepackage{cite}
\usepackage{url}

% Title and Authors
\title{Your Research Paper Title}
\author{
    \IEEEauthorblockN{First Author\IEEEauthorrefmark{1}, Second Author\IEEEauthorrefmark{2}, and Third Author\IEEEauthorrefmark{3}}
    \IEEEauthorblockA{\IEEEauthorrefmark{1}Department of Computer Science, University A\\ Email: first.author@universitya.edu}
    \IEEEauthorblockA{\IEEEauthorrefmark{2}Department of Computer Science, University B\\ Email: second.author@universityb.edu}
    \IEEEauthorblockA{\IEEEauthorrefmark{3}Department of Computer Science, University C\\ Email: third.author@universityc.edu}
}

\begin{document}

\maketitle

\begin{abstract}
   Lorem Ipsum is simply dummy text of the printing and typesetting industry. Lorem Ipsum has been the industry's standard dummy text ever since the 1500s, when an unknown printer took a galley of type and scrambled it to make a type specimen book. It has survived not only five centuries, but also the leap into electronic typesetting, remaining essentially unchanged. It was popularised in the 1960s with the release of Letraset sheets containing Lorem Ipsum passages, and more recently with desktop publishing software like Aldus PageMaker including versions of Lorem Ipsum
\end{abstract}

\section{Introduction}
Lorem Ipsum is simply dummy text of the printing and typesetting industry. Lorem Ipsum has been the industry's standard dummy text ever since the 1500s, when an unknown printer took a galley of type and scrambled it to make a type specimen book. It has survived not only five centuries, but also the leap into electronic typesetting, remaining essentially unchanged. It was popularised in the 1960s with the release of Letraset sheets containing Lorem Ipsum passages, and more recently with desktop publishing software like Aldus PageMaker including versions of Lorem Ipsum

\section{Inspiration}
Moths are fancy insects, which are highly similar to the family of butterflies. Basically, thereime: larvae and adult. The larvae is converted to moth in cocoons.
The most interesting fact about moths is their special naviga- tion methods in night. They have been evolved to fly in night using the moon light. They utilized a mechanism called transverse orien- tation for navigation. In this method, a moth flies by maintaining a fixed angle with respect to the moon, a very effective mechanism	

\section{Methodology}
Describe the methods and techniques used to conduct your research. This section should provide enough detail for others to replicate your study.

\section{Results}
Present the results of your research. Use figures and tables to help illustrate your findings.

\section{Discussion}
Interpret the results and discuss their implications. Compare your findings with previous works and discuss any limitations of your study.

\section{Conclusion}
Summarize the main findings of your research and suggest directions for future work.

\section*{Acknowledgments}
Acknowledge any support you received for your research, including funding, equipment, or assistance from colleagues.

\bibliographystyle{IEEEtran}
\bibliography{references}

\end{document}
