\documentclass[conference]{IEEEtran}

\usepackage{graphicx}
\usepackage{amsmath}
\usepackage{cite}
\usepackage{url}

% Title and Authors
\title{Enhanced Moth Flame Optimization}
\author{
    \IEEEauthorblockN{First Author\IEEEauthorrefmark{1}, Second Author\IEEEauthorrefmark{2}, and Third Author\IEEEauthorrefmark{3}}
    \IEEEauthorblockA{\IEEEauthorrefmark{1}Department of Computer Science, University A\\ Email: first.author@universitya.edu}
    \IEEEauthorblockA{\IEEEauthorrefmark{2}Department of Computer Science, University B\\ Email: second.author@universityb.edu}
    \IEEEauthorblockA{\IEEEauthorrefmark{3}Department of Computer Science, University C\\ Email: third.author@universityc.edu}
}

\begin{document}

\maketitle

\begin{abstract}
   Lorem Ipsum is simply dummy text of the printing and typesetting industry. Lorem Ipsum has been the industry's standard dummy text ever since the 1500s, when an unknown printer took a galley of type and scrambled it to make a type specimen book. It has survived not only five centuries, but also the leap into electronic typesetting, remaining essentially unchanged. It was popularised in the 1960s with the release of Letraset sheets containing Lorem Ipsum passages, and more recently with desktop publishing software like Aldus PageMaker including versions of Lorem Ipsum
\end{abstract}

\section{Introduction}
Global optimization problems are intricate and marked by diverse properties such as non-linearity, non-separability, symmetry, asymmetry, smoothness with narrow ridges, unimodality, multimodality, non-differentiability, and high dimensionality. These attributes pose challenges for current optimization algorithms, and finding the global optimum remains a long-standing objective in this field. To tackle these challenges, a range of metaheuristic algorithms have been introduced, showcasing impressive performance in exploring the problem space and approximating promising regions within reasonable timeframes.
Metaheuristic algorithms have been extensively applied in recent years to approximate near-optimal solutions for real-world problems across various domains, including discrete optimization, continuous optimization, and constrained engineering problems. These algorithms are categorized into local search-based algorithms and population-based algorithms. The primary advantage of local search methods is their fast search speeds, but their main drawback is a focus on exploitation over exploration, which increases the risk of getting trapped in local optima. Conversely, population-based algorithms, which consider a population of solutions simultaneously, recombine current solutions to generate new ones in each iteration. These methods are effective in identifying promising regions in the search space but less effective in exploiting the regions being explored.
Nature-inspired algorithms fall into three main categories: evolutionary, physics-based, and swarm intelligence (SI). Evolutionary algorithms draw inspiration from Darwin's theory of evolution, with examples including genetic algorithm (GA), genetic programming (GP), differential evolution (DE), evolution strategy (ES), and quantum-based avian navigation optimizer (QANA). Physics-based algorithms derive meaningful search strategies from physics and mathematical laws to address optimization problems, such as big bang–big crunch (BB-BC), charged system search (CSS), ray optimization (RO), and atom search optimization (ASO). Swarm intelligence-based methods are inspired by animal societies and social insect colonies, mimicking the behavior of swarming social insects, schools of fish, or flocks of birds. These methods are known for their flexibility and robustness.
The Moth-Flame Optimization (MFO) algorithm is a recent population-based metaheuristic developed by Mirjalili in 2015, modeled after the moths' navigation technique known as transverse orientation. Moths navigate at night using moonlight, maintaining a fixed angle to find their path. This behavior was formulated into an optimization technique combining a population-based algorithm with a local search strategy, enabling both global exploration and local exploitation. Like other metaheuristics, MFO is simple, flexible, and easy to implement, making it suitable for a wide range of problems. Due to these merits, MFO has been successfully applied to various optimization problems such as scheduling, inverse problems and parameter estimation, classification, economics, medical applications, power energy, and image processing.
In nature, over 160,000 different species of moths have been identified, which share a similar life cycle with butterflies, involving larva and adult stages, with the transformation occurring within cocoons. One of the most fascinating aspects of moths’ lives is their unique nocturnal navigation method. Moths have evolved to fly at night using moonlight, employing a mechanism known as transverse orientation for navigation. This mechanism enables a moth to maintain a constant angle relative to the moon, effectively facilitating long-distance travel in a straight line. Since the moon is distant, this method ensures straight-line flight. Humans can use a similar navigation technique: if the moon is in the southern sky and a person wants to head east, they can walk straight by keeping the moon to their left. However, moths often spiral around lights instead of traveling straight. This is because transverse orientation is effective only when the light source is very distant, like the moon. Moths are frequently observed circling artificial or natural point light sources until they become exhausted. This behavior arises from the inefficiency of transverse orientation when the light source is nearby. Artificial lights, such as lamps or candles, cause moths to spiral towards them because of the short distance, leading moths to maintain the same angle with the light source. Thus, in the presence of artificial lights, moths exhibit spiral flight paths around the lights.
The MFO algorithm has been widely used in physics-related fields. For instance, Hitarth Buch proposed an improved adaptive MFO algorithm to solve the optimal power flow problem, while Avishek Das used it to determine the optimal current excitation weights and spacing between array elements in a three-ring structure of the improved concentric antenna array (CCAA). Additionally, the MFO algorithm has applications in the medical field and beyond. For example, Gehad Ismail Sayed optimized the MFO population using neutrophils for automatic detection of mitosis in breast cancer tissue images, and Mingjing Wang applied a chaotic MFO algorithm to medical diagnosis, proposing a new learning scheme for a nuclear limit learning machine.
.
\section{Moth Flame optimizer}
Lorem Ipsum is simply dummy text of the printing and typesetting industry. Lorem Ipsum has been the industry's standard dummy text ever since the 1500s, when an unknown printer took a galley of type and scrambled it to make a type specimen book. It has survived not only five centuries, but also the leap into electronic typesetting, remaining essentially unchanged. It was popularised in the 1960s with the release of Letraset sheets containing Lorem Ipsum passages, and more recently with desktop publishing software like Aldus PageMaker including versions of Lorem Ipsum	

\section{Related Work}
Moth-Flame Optimization Algorithm: Variants and 
Applications
In the paper "Moth-Flame Optimization Algorithm: Variants 
and Applications," Shehab et al. [1] reviewed the MFO 
algorithm's applications and improvements. The original 
MFO algorithm was designed to simulate the navigation of 
moths using transverse orientation, but its initial version 
faced issues like premature convergence. To address these, 
the authors discussed various enhancements, including 
chaotic maps and opposition-based learning. These 
improvements were tested on several benchmark problems, 
showing significant performance gains. For instance, 
Improved Moth-Flame Optimization (IMFO)
In the follow-up study on Improved Moth-Flame 
Optimization (IMFO), the same group of authors [2] 
focused on refining the original MFO algorithm to tackle its 
limitations more effectively. They introduced features such 
as opposition-based learning and adaptive parameter control 
to enhance the algorithm's robustness. This was tested on 
complex optimization problems where the IMFO 
demonstrated better performance in terms of convergence 
speed and solution accuracy compared to the standard 
MFO. These enhancements helped the algorithm avoid local 
optima by diversifying the search process and dynamically 
adjusting control parameters based on the problem's 
landscape​​.
Symmetry in Optimization Algorithms
In another significant contribution, the paper "Symmetry in 
Optimization Algorithms" by the same authors [3] explored 
the role of symmetry in enhancing the MFO algorithm's 
efficiency. The study leveraged symmetrical properties in 
problem spaces to simplify the search process and improve 
optimization results. By integrating these symmetrical 
considerations, the MFO algorithm's search space was 
effectively reduced, leading to quicker convergence and 
higher-quality solutions. This approach was particularly 
beneficial in applications like image processing and 
engineering design, where symmetry is prevalent, 
demonstrating that understanding and utilizing problemspecific properties can significantly boost optimization 
performance​​.
Enhanced Hybrid Moth-Flame Optimization Algorithm 
Using New Selection Schemes
In the paper "Enhanced Hybrid Moth-Flame Optimization 
Algorithm Using New Selection Schemes," Shehab et al. 
introduced novel selection schemes to improve the MFO 
algorithm's performance. The methodology involved 
hybridizing MFO with other optimization techniques to 
enhance both exploration and exploitation phases. By 
integrating adaptive parameter control and diverse 
initialization strategies, the authors aimed to maintain 
population diversity and prevent premature convergence. 
The enhanced MFO algorithm was tested on various 
benchmark problems and real-world applications, showing 
significant improvements in convergence speed and solution 
accuracy compared to the standard MFO. However, despite 
these advancements, the study acknowledged limitations 
such as increased computational complexity and challenges 
in highly dynamic environments.
Moth-Flame Optimization: Theory, Modifications, 
Hybridizations, and Applications
The comprehensive review "Moth-Flame Optimization: 
Theory, Modifications, Hybridizations, and Applications" 
by Sahoo et al. explored the theoretical foundations, various 
modifications, hybridizations, and applications of the MFO 
algorithm. The authors provided a systematic categorization 
of these modifications and applications, highlighting their 
strengths and weaknesses. This review covered adaptive 
parameter settings and how these modifications were 
applied in fields like chemical engineering, medical 
diagnostics, and image processing. The review underscored 
the enhanced performance of the MFO algorithm in terms 
of accuracy and convergence speed when hybridized with 
techniques like genetic algorithms and particle swarm 
optimization. Nonetheless, the study identified persistent 
issues such as susceptibility to local optima and inconsistent 
performance across different problem types, suggesting a 
need for more adaptive and problem-specific versions of the 
MFO.
MFO-SFR: An Enhanced Moth-Flame Optimization 
Algorithm
The paper "MFO-SFR: An Enhanced Moth-Flame 
Optimization Algorithm" presented a new selection 
mechanism designed to improve the MFO algorithm's 
balance between exploration and exploitation. The 
methodology included rigorous evaluations using CEC 
2018 benchmark test functions and real-world mechanical 
engineering problems from the CEC 2020 test suite. The 
MFO-SFR algorithm demonstrated superior performance in 
these tests, achieving improved convergence rates and 
higher-quality solutions compared to the standard MFO. 
This confirmed the algorithm's practical utility and 
robustness in diverse optimization scenarios. However, the 
study also noted limitations, including increased 
computational demands and implementation complexity, 
suggesting that further refinements are needed to optimize 
performance in highly complex and dynamic environment
\bibliographystyle{IEEEtran}
\bibliography{references}

\end{document}
